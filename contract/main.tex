\documentclass{article}
\usepackage[utf8]{inputenc}
\usepackage{spreadtab}

\title{Book Repair}
\author{Ben Greene}
\date{\today}

\begin{document}

\maketitle

\section{Overview}

I am trained in book preservation. I acquired most of my skills from Terry Kissner,
the head of Preservation and Special Collections at Carleton College. I am not trained
in conservation. The difference between preservation and conservation is, for the most
part, the goal. By preserving a book, I make it usable by the reader for longer. Often
the repairs I do are irreversible and can decrease the market value of the volume. The
purpose of conservation, for the most part, is to keep the book intact and historically
accurate as possible. This means that all repairs done in book conservation must be
entirely reversible. This often means that books that have been conserved?  are still 
unusable by the reader. If I receive a book that I feel is too valuable to be invasively repaired with my skill set, I will recommend non-irreversible, protective services, such as jacketing or box storage. 

Books are very important to me, and I want to help you be able to use and 
protect your books to the very best of my ability. I hope that I can help you.

Please look over the list of services I can provide:
\begin{itemize}
\item Re-backing

This is good for books whose spines are broken or detached, but
whose covers, boards and text blocks are in good condition. Please note
that this process is irreversible and is not recommended for older volumes
because it greatly reduces their value. That being said, re-backing a broken book makes it more usable.

\item Casing Paperbacks

This process turns paperback books into hardcover volumes. This is great for protecting
a book that you know you will be using often in the future.

\item Hinge Repair

This process can be applied in addition to re-backing and reinforces where the text of the book
connects to the cover. This is not necessary unless the hinge is broken or the text block
of the book is incredibly heavy. This is also an irreversible process, so this process 
is used with discretion.

\item Pamphlet Binding

This process is great for protecting stapled pamphlets with one signature like manuals, smaller
treatises, or periodical magazines.

\item Jacketing

Jacketing is a great way to preserve items that are too old or delicate to be preserved using
the above methods. A jacket is another cover, usually made of polyester that protects the book
from dust, scrapes, and other minor damage. It is non-invasive and will not decrease the value
of the book.

\item Boxes

Boxes are the strongest method of protecting materials that I provide. They range from simple
archival quality board boxes, which are great for storing and organizing entire series of 
periodicals, to felt padded clamshell boxes, which insulate and provide extremely robust
protection for the book. All boxes are custom made and measured to fit.


\end{itemize}

\section{Payment}

When you pay me to repair a book, you are paying for labor and material. I charge an hourly rate and add the material charge. My hourly rate is \$15, and material cost is typically from \$2 to \$10.

\section{Your Agreement}
By signing your name below, you affirm that you have read the above text and know that any changes I make to your book will extend its useful lifetime, but are irreversible and may decrease its monetary or collectable value.

That being said, processes like jacketing and box making in no way alter the condition of your book and help maintain its value.

\vspace{10mm}
Signature
\vspace{5mm}
\noindent\rule{8cm}{0.4pt}

\vspace{5mm}
Date 
\vspace{5mm}
\noindent\rule{8cm}{0.4pt}
\end{document}
